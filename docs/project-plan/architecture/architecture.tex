%!TEX program=xelatex
%!TEX spellcheck = en_GB
\documentclass[final]{report}
\input{../../.library/preamble.tex}
\input{../../.library/style.tex}
\addbibresource{../../.library/bibliography.bib}
\begin{document}
\chapter{Architecture}
\label{ch:architecture}
The system will be divided in several separate entities, in \cref{fig:system-architecture} a simple schematic view is shown. The names used for interactions in the diagram are to give a general idea of communications and are used with a different name in the model.

\begin{figure}[H]
	\centering
	\subimport{resources/}{system-architecture.tikz}
	\caption{System Architecture}
	\label{fig:system-architecture}
\end{figure}

\section{Track Sensor Controller}\label{sec:architecture-sensor-controller}
Each Track has a Track Sensor Controller to interface with the sensors $S_{n,w}$, $S_{n,m}$, $S_{n,e}$ (where $n = 1,2,3,..$ for the number of the track).
This unit determines if there is a train on the track and whenever it is entering or leaving.
It has one boolean output signal, $train\_on\_track$ (or abbreviated to $T$).


\section{Sensor Aggregator}\label{sec:architecture-sensor-aggregator}
Their outputs will be put through an OR gate resulting in the signal $train\_detected$.
%TODO[c]: write some text about the agg here

\section{Light Controller}\label{sec:architecture-light-controller}
This unit will control the lights.
It will orchestrate the blink controllers and communicate it's own status toward the Bell Controller, using signal $light\_status$.

One unit of this type will be in the system.

\section{Bell Controller}\label{sec:architecture-bell-controller}
This unit directly controls the bells.
It gets information from both the Barrier Controller, Light Controller and Track Sensor Controllers.
It sends it's own status using $bell\_status$ toward the Light Controller and the Barrier Controller.

One unit of this type will be in the system.

\section{Barrier Controller}\label{sec:architecture-barrier-controller}
This unit will control the barriers opening and closing.
It will communicate with the Track Sensor Controller to see if there is any reason to change it's state, it mainly reacts to $train\_detected$.
It will signal it's own state to the Bell Controller, using signal $barrier\_status$.

One unit of this type will be in the system.

\end{document}