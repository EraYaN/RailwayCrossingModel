\documentclass[final]{report}
\input{../../.library/preamble.tex}
\input{../../.library/style.tex}
\addbibresource{../../.library/bibliography.bib}

\begin{document}
\chapter{Translated Requirements}
After having described the system and it's requirements in natural language it's time to formulate them as $\mu$-calculus formulae that can be used for verification. In \cref{ch:interactions} all the actions in the model that will be used in the formulae in this chapter were described.

\section{Sensors}
In \cref{tab:sensors_transitions} the sensors' states and the consequent \texttt{    train} or \texttt{no\_train} output signal are displayed when a train is moving from East to West and from West to East.

\begin{table}[H]
\centering
    \begin{tabular}{|c|c|c|c|c|}
    \hline 
    $S_{w}$ & $S_{m}$ & $S_{e}$ & East to West       & West to East       \\ \hline 
    OFF     & OFF     & OFF     & \texttt{no\_train} & \texttt{no\_train} \\ \hline 
    ON      & OFF     & OFF     & \texttt{    train} & \texttt{no\_train} \\ \hline 
    OFF     & OFF     & OFF     & \texttt{    train} & \texttt{no\_train} \\ \hline 
    OFF     & ON      & OFF     & \texttt{    train} & \texttt{    train} \\ \hline 
    OFF     & OFF     & OFF     & \texttt{no\_train} & \texttt{    train} \\ \hline 
    OFF     & OFF     & ON      & \texttt{no\_train} & \texttt{    train} \\ \hline 
    OFF     & OFF     & OFF     & \texttt{no\_train} & \texttt{no\_train} \\ \hline 
    \end{tabular}
    \caption{Sensors states and sensor controller output signal for a train moving in either possible direction.}
    \label{tab:sensors_transitions}
\end{table}

Based on this, the following translated requirements may be formulated: 

\begin{enumerate}
\item There cannot be a \texttt{no\_train} signal after a train has triggered an ON in any outer sensor for the first time (i.e a train has entered the tracks) and has not triggered an 
an OFF in the middle sensor:\\

\item There cannot be a \texttt{train} signal after a train has triggered an OFF in the middle sensor and has not triggered an OFF in any outer sensor (i.e. a train is leaving the
tracks):\\

\item There cannot be a \texttt{train} signal if after a train has triggered an OFF in any outer sensor and not triggered an ON in the middle sensor (i.e a train has exit
 the tracks), another train does not trigger an ON in any outer sensor:\\

\end{enumerate}


\section{Lights}

\begin{enumerate}
    \item The lights cannot turn on when there is no train on the tracks:\\
        \texttt{[true*][tx\_sensors\_agg(no\_train).\notbar{tx\_sensors\_agg(train)}*.turnon\_lights]false}

    \item The lights cannot turn off as long as there is a train on the tracks:\\
        \texttt{[true*][tx\_sensors\_agg(train).\notbar{tx\_sensors\_agg(no\_train)}*.turnoff\_lights]false}

    \item The lights cannot turn off as long as the bells are on:\\
        \texttt{[true*][turnon\_bells.\notbar{turnoff\_bells}*.turnoff\_lights]false}
\end{enumerate}


\section{Bells}

\begin{enumerate}
    \item The bells cannot turn on if the lights are off:\\
        \texttt{[true*][turnoff\_lights.\notbar{turnon\_lights}*.turnoff\_bells]false}

    \item The bells cannot turn on if there is no train on the track:\\
        \texttt{[true*][tx\_sensors\_agg(no\_train).\notbar{tx\_sensors\_agg(train)}*.turnon\_bells]false}

    \item The bells cannot turn off if the barriers are lowered:\\
        \texttt{[true*][lower\_barriers.\notbar{raise\_barriers}*.turnoff\_bells]false}

    \item The bells cannot turn off if there is a train on the track:\\
        \texttt{[true*][tx\_sensors\_agg(train).\notbar{tx\_sensors\_agg(no\_train)}*.turnoff\_bells]false}
\end{enumerate}


\section{Barriers}

\begin{enumerate}
    \item The barriers cannot be raised when there is a train on the tracks:\\
        \texttt{[true*][tx\_sensors\_agg(train).\notbar{tx\_sensors\_agg(no\_train)}*.raise\_barriers]false}

    \item The barriers cannot be lowered if there is no train on the tracks:\\
        \texttt{[true*][tx\_sensors\_agg(no\_train).\notbar{tx\_sensors\_agg(train)}*.lower\_barriers]false}

    \item The barriers cannot be lowered if the bells are off:\\
        \texttt{[true*][turnoff\_bells.\notbar{turnon\_bells}*.lower\_barriers]false}
\end{enumerate}


\section{General}

\begin{enumerate}
    \item From every state, the crossing needs to be able to be closed (barriers lowered, lights and bells on):\\
        \texttt{[true*]<true*.lower\_barriers>true}

    \item From every state, the crossing needs to be able to be opened(lights and bells off, barriers raised):\\
        \texttt{[true*]<true*.turnoff\_lights>true}
\end{enumerate}

\end{document}