%!TEX program=xelatex
%!TEX spellcheck = en_GB
\documentclass[final]{report}
\input{../../.library/preamble.tex}
\input{../../.library/style.tex}
\addbibresource{../../.library/bibliography.bib}
\begin{document}
%TODO[L]: make the tables fit, update trainon/off and implement broadcast signals.
\chapter{Interactions}
\label{ch:interactions}
This system implements five parallel components that interact with each other and the environment. In this section the interactions of the different components are defined per component and divided in input and output signals.

\section{Sensor controllers}
\begin{table}[H]
\centering
\begin{tabular}{|l|l|l|}
\hline
set\_sensors(s, state) & s=s1, s2, s3; state=on, off & Gives state of sensors to the sensor controller   \\\hline
tx\_sensors(Train) & Train=train, no\_train & Transmits presence of train to the sensor aggregator.  \\\hline
\end{tabular}
\caption{Interactions of the sensor controller}
\label{tb:sensorSignals}
\end{table}

\cref{tb:sensorSignals} shows the interactions of the sensor controller with the sensor aggregator and the sensors. The only job of the individual sensor controllers is to tell the sensor aggregator if there is a train on \textit{their} track so the aggregator can then send a signal to the other controllers if there is a train on \textit{any} track. 

\section{Sensor aggregator}
\begin{table}[H]
\centering
\begin{tabular}{|l|l|l|}
\hline
rx\_sensors(t, Train) & t=t1, t2; Train=train, no\_train & Receives status of tracks from the sensor controllers  \\\hline
tx\_sensors\_agg(Train) & Train=train, no\_train & Transmits status of tracks to the other controllers.  \\\hline
\end{tabular}
\caption{Interactions of the sensor aggregator}
\label{tb:sensorAggSignals}
\end{table}

In \cref{tb:sensorAggSignals} the interactions of the sensor aggregator with the other controllers are shown. Only two signals are needed for this component to do it's job. It doesn't need to communicate with any external components.

\section{Lights controller}
\begin{table}[H]
\centering
\begin{tabular}{|l|l|l|}
\hline
rx\_sensors\_agg(Train)  & Train=train, no\_train & Receives status of tracks from sensor controller \\
rx\_bells(state)   & state=on, off          & Receives state of bells from bells controller               \\ \hline
turnon\_lights &         & Turns on lights                                     \\
turnoff\_lights & & Turns off lights\\
tx\_Lights(state)  & state=on, off          & Transmits state of lights to bells controller               \\ \hline
\end{tabular}
\caption{Interactions of the lights controller}
\label{tb:lightsSignals}
\end{table}

In \cref{tb:lightsSignals} the interactions of the lights controller with the other controllers and the lights are shown. Most of the interactions are communications with the other controllers with the actions \textit{turnon\_lights} and \textit{turnoff\_lights} being responsible for actually controlling the lights.

\section{Bells controller}
\begin{table}[H]
\centering
\begin{tabular}{|l|l|l|}
\hline
rx\_sensors\_agg(Train)   & Train=train, no\_train & Receives status of tracks from sensor controller           \\
rx\_lights(state)   & state=on, off          & Receives state of lights from lights controller                      \\
rx\_barriers(state) & state=on, off          & Receives state of barriers from barriers controller                  \\ \hline
turnon\_bells   &          & Turns bells on                                              \\
turnoff\_bells   &          & Turns bells off                                              \\
tx\_bells(state)    & state=on, off          & Transmits state of bells to bells controller and barriers controller\\ \hline
\end{tabular}
\caption{Interactions of the bells controller}
\label{tb:bellsSignals}
\end{table}

The interactions of the bells controller shown in \cref{tb:bellsSignals} look a lot like those of the lights controller and require no further explanation.

\section{Barriers controller}
\begin{table}[H]
\centering
\begin{tabular}{|l|l|l|}
\hline
rx\_sensors\_agg(Train)    & Train=train, no\_train & Receives status of tracks from sensor controller \\
rx\_bells(state)     & state=on, off          & Receives state of bells from bells controller               \\ \hline
lower\_barriers &  & Lowers the barriers                \\
raise\_barriers &  & Raises the barriers                \\
tx\_barriers(state)  & state=on, off          & Transmits state of barriers to bells controller             \\ \hline
\end{tabular}
\caption{Interactions of the bells controller}
\label{tb:barriersSignals}
\end{table}

Once again the interactions show in \cref{tb:barriersSignals} follow the same template as earlier discussed controllers and require no further explanation.

\section{Communications}
\begin{table}[H]
\centering
\begin{tabular}{|l|l|l|}
\hline
rx\_sensors(Train)|tx\_sensors(Train)     & Train=train, no\_train & comm\_sensors(Train)   \\
rx\_sensors\_agg(Train)|tx\_sensors\_agg(Train)     & Train=train, no\_train & comm\_sensors\_agg(Train)   \\
rx\_lights(state)|tx\_lights(state)     & state=on, off          & comm\_lights(state)   \\
rx\_bells(state)|tx\_bells(state)       & state=on, off          & comm\_bells(state)    \\
rx\_barriers(state)|tx\_barriers(state) & state=on, off          & comm\_barriers(state) \\ \hline
\end{tabular}
\caption{Communications in the system}
\label{tb:communicationsTable}
\end{table}

\cref{tb:communicationsTable} shows the communications defined in the model.

\end{document}