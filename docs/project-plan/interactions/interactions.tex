%!TEX program=xelatex
%!TEX spellcheck = en_GB
\documentclass[final]{report}
\input{../../.library/preamble.tex}
\input{../../.library/style.tex}
\addbibresource{../../.library/bibliography.bib}
\begin{document}
\chapter{Interactions}\label{ch:interactions}
This system implements five parallel components that interact with each other and the environment.
In this section the interactions of the different components are defined per component and divided in input and output signals.

\section{Type Definitions}\label{sec:interactions-typedef}
All the system components make use of a couple type definitions.
The \texttt{sort} structures used throughout the report are described below.

To make a distinction between the $n$ tracks of \cref{fig:system-architecture}, the a \texttt{TrackID} structure with the possible values of \texttt{t1} and \texttt{t2}.
For the three sensors of a track, a similar structure is needed: \texttt{SensorID} with the possible values \texttt{s\_w}, \texttt{s\_m} and \texttt{s\_e}.

Most of the subsystems control some external peripheral which has either an on, or off state.
To model this, the \texttt{State} structure is defined being either \texttt{on} or \texttt{off}.

The last structure that is used throughout the model is \texttt{Train} which is used to indicate whether there is currently a train on the track or not (\texttt{train}, \texttt{no\_train}).

\section{Interactions}\label{sec:interactions-external}
\subsection{Incoming}
Each track has its on set of three sensors, each of the sensors can do \texttt{set\_sensors} action, this actions has three parameters being \texttt{TrackID}, \texttt{SensorID} and \texttt{State}.

\subsection{Outgoing}
The outgoing actions for each of the three controllers are showed in the table below, they are self explanatory.

\begin{table}[H]
\centering
    \begin{tabular}{|l|l|l|}
    \hline
\nameref{sec:architecture-light-controller} & \nameref{sec:architecture-bell-controller} & \nameref{sec:architecture-barrier-controller} \\ \hline
\texttt{turnon\_light1}                     & \texttt{turnon\_bell1}                     & \texttt{lower\_barrier1} \\
\texttt{turnon\_light2}                     & \texttt{turnon\_bell2}                     & \texttt{lower\_barrier2} \\
\texttt{turnoff\_light1}                    & \texttt{turnoff\_bell1}                    & \texttt{raise\_barrier1} \\
\texttt{turnoff\_light2}                    & \texttt{turnoff\_bell2}                    & \texttt{raise\_barrier2} \\ \hline
    \end{tabular}
    \caption{Interactions of the sensor controller}
    \label{tab:sensorSignals}
\end{table}

\subsection{Error}
In our model there are also some error states, these are modeled by first doing an \texttt{error} action and then entering the deadlock state by doing the action \texttt{delta}.


\subsection{Internal Communication}


\end{document}