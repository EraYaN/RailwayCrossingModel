%!TEX program=xelatex
%!TEX spellcheck = en_GB
\documentclass[final]{report}
\input{../../.library/preamble.tex}
\input{../../.library/style.tex}
\addbibresource{../../.library/bibliography.bib}
\begin{document}
\chapter{Interactions}
This system implements five parallel components that interact with each other and the environment. In this section the input and output signals of the different components are defined.

\section{Sensor controller}
\begin{table}[h]
\centering
\begin{tabular}{|l|l|l|}
\hline
OutSensor1(t) & t=T1, T2 & Gives state of one of the outer sensors to the sensor controller  \\
MiddleSensor(t) & t=T1, T2           & Gives state of middle
 sensors to the sensor controller \\ 
OutSensor2(t) & t=T1, T2 & Gives state of one of the outer sensors to the sensor controller  \\\hline
tx\_sensors(Train) & Train=train, no\_train & Transmits presence of train on track to the other controllers.  \\\hline
\end{tabular}
\caption{Interactions of the sensor controller}
\label{sensorSignals}
\end{table}

\cref{sensorSignals} shows the interactions of the sensorcontroller with other controllers and the sensors. In the interactions "t" defines on what track the sensor sending the signal is.

\section{Lights controller}
\begin{table}[h]
\centering
\begin{tabular}{|l|l|l|}
\hline
rx\_Sensor(Train)  & Train=train, no\_train & Receives presence of train on tracks from sensor controller \\
rx\_Bells(state)   & state=on, off          & Receives state of bells from bells controller               \\ \hline
act\_Lights(state) & state=on, off          & Turns lights on or off                                      \\
tx\_Lights(state)  & state=on, off          & Transmits state of lights to bells controller               \\ \hline
\end{tabular}
\caption{Interactions of the lights controller}
\label{lightsSignals}
\end{table}

Todo

\section{Bells controller}
\begin{table}[h]
\centering
\begin{tabular}{|l|l|l|}
\hline
rx\_Sensor(Train)   & Train=train, no\_train & Receives presence of train on tracks from sensor controller           \\
rx\_Lights(state)   & state=on, off          & Receives state of lights from lights controller                      \\
rx\_Barriers(state) & state=on, off          & Receives state of barriers from barriers controller                  \\ \hline
act\_Bells(state)   & state=on, off          & Turns bells on or off                                                \\
tx\_Bells(state)    & state=on, off          & Transmits state of bells to bells controller and barriers controller\\ \hline
\end{tabular}
\caption{Interactions of the bells controller}
\label{bellsSignals}
\end{table}

Todo

\section{Barriers controller}
\begin{table}[h]
\centering
\begin{tabular}{|l|l|l|}
\hline
rx\_Sensor(Train)    & Train=train, no\_train & Receives presence of train on tracks from sensor controller \\
rx\_Bells(state)     & state=on, off          & Receives state of bells from bells controller               \\ \hline
act\_Barriers(state) & state=on, off          & Closes or opens barrier(Todo, define states)                \\
tx\_Barriers(state)  & state=on, off          & Transmits state of barriers to bells controller             \\ \hline
\end{tabular}
\caption{Interactions of the bells controller}
\label{barriersSignals}
\end{table}

Todo

\section{Communications}
\begin{table}[h]
\centering
\begin{tabular}{|l|l|l|}
\hline
rx\_Sensor(Train)|tx\_Sensor(Train)     & Train=train, no\_train & comm\_Sensor(Train)   \\
rx\_Lights(state)|tx\_Lights(state)     & state=on, off          & comm\_Lights(state)   \\
rx\_Bells(state)|tx\_Bells(state)       & state=on, off          & comm\_Bells(state)    \\
rx\_Barriers(state)|tx\_Barriers(state) & state=on, off          & comm\_Barriers(state) \\ \hline
\end{tabular}
\caption{Communications in the system}
\label{communicationsTable}
\end{table}

todo

\end{document}