%!TEX program=xelatex
%!TEX spellcheck = en_GB
\documentclass[final]{report}
% Include all project wide packages here.
%\usepackage{fullpage}
\usepackage[a4paper,margin=2.5cm,top=2cm]{geometry}
\usepackage{polyglossia}
\setmainlanguage{english}
\usepackage{csquotes}
\usepackage{graphicx}
\usepackage{pdfpages}
\usepackage{caption}
\usepackage[list=true]{subcaption}
\usepackage{float}
\usepackage{standalone}
\usepackage{import}
\usepackage{tocloft}
\usepackage{wrapfig}
\usepackage{authblk}
\usepackage{array}
\usepackage{booktabs}
\usepackage[title,titletoc]{appendix}
\usepackage{fontspec}
\usepackage{pgfplots}
\usepackage{tikz}
\usepackage{siunitx}
\usepackage{units}
\usepackage{amsmath}
\usepackage{mathtools}
\usepackage{unicode-math}
\usepackage{rotating}
\usepackage{titlesec}
\usepackage{titletoc}
\usepackage{blindtext}
\usepackage{color}
\usepackage{enumitem}
\usepackage{tabularx}
\usepackage{titling}
\usepackage[%
siunitx,
fulldiodes,
europeanvoltages,
europeancurrents,
europeanresistors,
americaninductors,
smartlabels]{circuitikz}

\usetikzlibrary{calc}
\usetikzlibrary{positioning}

\pgfplotsset{compat=newest}
\pgfplotsset{plot coordinates/math parser=false}
\usetikzlibrary{plotmarks}
\usepgfplotslibrary{patchplots}
\newlength\figureheight
\newlength\figurewidth

\usepackage[
%backend=bibtex,
backend=biber,
	texencoding=utf8,
bibencoding=utf8,
style=numeric,
citestyle=numeric,
    sortlocale=en_US,
    language=auto,
    backref=true,
    abbreviate=false,
    date=iso8601
]{biblatex}


\usepackage{listings}
\newcommand{\includecode}[4][c]{\lstinputlisting[caption=#2, escapechar=, style=#1,label=#4]{#3}}
\newcommand{\superscript}[1]{\ensuremath{^{\textrm{#1}}}}
\newcommand{\subscript}[1]{\ensuremath{_{\textrm{#1}}}}


\newcommand{\chapternumber}{\thechapter}
\renewcommand{\appendixname}{Appendix}
\renewcommand{\appendixtocname}{Appendices}
\renewcommand{\appendixpagename}{Appendices}


\setlist[enumerate]{labelsep=*, leftmargin=1.5pc}
\setlist[enumerate,1]{label=\arabic*., ref=\arabic*}
\setlist[enumerate,2]{label=\arabic*.,ref=\theenumi.\arabic*}
\setlist[enumerate,3]{label=\arabic*., ref=\theenumii.\arabic*}

\usepackage{xr-hyper}
\usepackage[hidelinks]{hyperref} %<--------ALTIJD ALS LAATSTE
\usepackage[nameinlink,noabbrev,capitalise]{cleveref} %<------- Clever Ref moet na hyperref
\crefname{app}{Appendix}{Appendices}
%\renewcommand{\familydefault}{\sfdefault}


\setmainfont{Myriad Pro}[Ligatures={Common,TeX}]
%\setmathfont{Asana Math}
\setmathfont{Asana-Math.otf}
\setmonofont[Scale=0.9]{Lucida Console}
\newfontfamily\headingfont{Minion Pro}[Ligatures={Common,TeX}]


%Design colors
\definecolor{accent1}{RGB}{0,100,200}
\definecolor{accent2}{RGB}{0,50,100}
\definecolor{tu-cyan}{RGB}{0,166,214}

\newcommand{\hsp}{\hspace{20pt}}
\titleformat{\chapter}[hang]{\Huge\headingfont}{\chapternumber\hsp\textcolor{accent2}{|}\hsp}{0pt}{\Huge\headingfont}

\titleformat{name=\chapter,numberless}[hang]{\Huge\headingfont}{\hsp\textcolor{accent2}{|}\hsp}{0pt}{\Huge\headingfont}

\titleformat{\section}[block]{\LARGE\headingfont}{\arabic{chapter}.\arabic{section}}{0.4em}{}
\titleformat{\subsection}[block]{\Large\headingfont}{\arabic{chapter}.\arabic{section}.\arabic{subsection}}{0.4em}{}
\titleformat{\subsubsection}[block]{\large\headingfont}{\arabic{chapter}.\arabic{section}.\arabic{subsection}.\arabic{subsubsection}}{0.4em}{}
\renewcommand{\arraystretch}{1.2}
\renewcommand{\baselinestretch}{1.25} 

\renewcommand\cfttoctitlefont{\headingfont\Huge}
\renewcommand\cftloftitlefont{\headingfont\Huge}
\renewcommand\cftlottitlefont{\headingfont\Huge}
\setcounter{lofdepth}{2}
\setcounter{lotdepth}{2}


\setlength{\parindent}{0pt}
\setlength{\parskip}{1em}


%SIuntix settings:
%default: 0V to 10V
%custom: 0 - 10V
\sisetup{range-phrase=--}
\sisetup{range-units=single}
\DeclareSIUnit\years{years}

%For code listings
\definecolor{black}{rgb}{0,0,0}
\definecolor{browntags}{rgb}{0.65,0.1,0.1}
\definecolor{bluestrings}{rgb}{0,0,1}
\definecolor{graycomments}{rgb}{0.4,0.4,0.4}
\definecolor{redkeywords}{rgb}{1,0,0}
\definecolor{bluekeywords}{rgb}{0.13,0.13,0.8}
\definecolor{greencomments}{rgb}{0,0.5,0}
\definecolor{redstrings}{rgb}{0.9,0,0}
\definecolor{purpleidentifiers}{rgb}{0.01,0,0.01}


\lstdefinestyle{csharp}{
language=[Sharp]C,
showspaces=false,
showtabs=false,
breaklines=true,
showstringspaces=false,
breakatwhitespace=true,
escapeinside={(*@}{@*)},
columns=fullflexible,
commentstyle=\color{greencomments},
keywordstyle=\color{bluekeywords}\bfseries,
stringstyle=\color{redstrings},
identifierstyle=\color{purpleidentifiers},
basicstyle=\ttfamily\small}

\lstdefinestyle{c}{
language=C,
showspaces=false,
showtabs=false,
breaklines=true,
showstringspaces=false,
breakatwhitespace=true,
escapeinside={(*@}{@*)},
columns=fullflexible,
commentstyle=\color{greencomments},
keywordstyle=\color{bluekeywords}\bfseries,
stringstyle=\color{redstrings},
identifierstyle=\color{purpleidentifiers},
}

\lstdefinestyle{matlab}{
language=Matlab,
showspaces=false,
showtabs=false,
breaklines=true,
showstringspaces=false,
breakatwhitespace=true,
escapeinside={(*@}{@*)},
columns=fullflexible,
commentstyle=\color{greencomments},
keywordstyle=\color{bluekeywords}\bfseries,
stringstyle=\color{redstrings},
identifierstyle=\color{purpleidentifiers}
}

\lstdefinestyle{vhdl}{
language=VHDL,
showspaces=false,
showtabs=false,
breaklines=true,
showstringspaces=false,
breakatwhitespace=true,
escapeinside={(*@}{@*)},
columns=fullflexible,
commentstyle=\color{greencomments},
keywordstyle=\color{bluekeywords}\bfseries,
stringstyle=\color{redstrings},
identifierstyle=\color{purpleidentifiers}
}

\lstdefinestyle{xaml}{
language=XML,
showspaces=false,
showtabs=false,
breaklines=true,
showstringspaces=false,
breakatwhitespace=true,
escapeinside={(*@}{@*)},
columns=fullflexible,
commentstyle=\color{greencomments},
keywordstyle=\color{redkeywords},
stringstyle=\color{bluestrings},
tagstyle=\color{browntags},
morestring=[b]",
  morecomment=[s]{<?}{?>},
  morekeywords={xmlns,version,typex:AsyncRecords,x:Arguments,x:Boolean,x:Byte,x:Char,x:Class,x:ClassAttributes,x:ClassModifier,x:Code,x:ConnectionId,x:Decimal,x:Double,x:FactoryMethod,x:FieldModifier,x:Int16,x:Int32,x:Int64,x:Key,x:Members,x:Name,x:Object,x:Property,x:Shared,x:Single,x:String,x:Subclass,x:SynchronousMode,x:TimeSpan,x:TypeArguments,x:Uid,x:Uri,x:XData,Grid.Column,Grid.ColumnSpan,Click,ClipToBounds,Content,DropDownOpened,FontSize,Foreground,Header,Height,HorizontalAlignment,HorizontalContentAlignment,IsCancel,IsDefault,IsEnabled,IsSelected,Margin,MinHeight,MinWidth,Padding,SnapsToDevicePixels,Target,TextWrapping,Title,VerticalAlignment,VerticalContentAlignment,Width,WindowStartupLocation,Binding,Mode,OneWay,xmlns:x}
}

\lstdefinestyle{python}{
language=Python,
showspaces=false,
showtabs=false,
breaklines=true,
showstringspaces=false,
breakatwhitespace=true,
escapeinside={(*@}{@*)},
columns=fullflexible,
commentstyle=\color{greencomments},
keywordstyle=\color{bluekeywords}\bfseries,
stringstyle=\color{redstrings},
identifierstyle=\color{purpleidentifiers},
}

%defaults
\lstset{
basicstyle=\ttfamily\scriptsize ,
extendedchars=false,
numbers=left,
numberstyle=\ttfamily\tiny,
stepnumber=1,
tabsize=4,
numbersep=5pt
}
\addbibresource{../../.library/bibliography.bib}
\begin{document}
%TODO[L]: make the tables fit, update trainon/off and implement broadcast signals.
\chapter{Interactions}
\label{ch:interactions}
This system implements five parallel components that interact with each other and the environment.
In this section the interactions of the different components are defined per component and divided in input and output signals.

\section{Sensor Controllers}
\begin{table}[H]
\centering
    \begin{tabular}{|l|l|l|}
    \hline
    \texttt{set\_sensors(s, state)} & \texttt{s=s1, s2, s3; state=on, off}  & Gives state of sensors to the sensor controller      \\ \hline
    \texttt{tx\_sensors(Train)}     & \texttt{Train=train, no\_train}       & Transmits presence of train to the sensor aggregator \\ \hline
    \end{tabular}
    \caption{Interactions of the sensor controller}
    \label{tab:sensorSignals}
\end{table}

\cref{tab:sensorSignals} shows the interactions of the sensor controller with the sensor aggregator and the sensors.
The only job of the individual sensor controllers is to tell the sensor aggregator if there is a train on \textit{their} track so the aggregator can then send a signal to the other controllers if there is a train on \textit{any} track.

\section{Sensor Aggregator}
\begin{table}[H]
\centering
    \begin{tabular}{|l|l|l|}
    \hline
    \texttt{rx\_sensors(t, Train)}   & \texttt{t=t1, t2; Train=train, no\_train} & Receives status of tracks from the sensor controllers \\ \hline
    \texttt{tx\_sensors\_agg(Train)} & \texttt{Train=train, no\_train}           & Transmits status of tracks to the other controllers   \\
    \texttt{rx\_sensors(t, Train)}   & \texttt{t=t1, t2; Train=train, no\_train} & Receives status of tracks from the sensor controllers \\ \hline
    \texttt{tx\_sensors\_agg(Train)} & \texttt{Train=train, no\_train}           & Transmits status of tracks to the other controllers   \\ \hline
    \end{tabular}
\caption{Interactions of the sensor aggregator}
\label{tab:sensorAggSignals}
\end{table}

In \cref{tab:sensorAggSignals} the interactions of the sensor aggregator with the other controllers are shown.
Apart from the normal communication signals the presence of two debug signals should be noted.
These were necessary to verify the requirements of the sensor controller.

\section{Light Controller}
\begin{table}[H]
\centering
    \begin{tabular}{|l|l|l|}
    \hline
    \texttt{rx\_sensors\_agg(Train)} & \texttt{Train=train, no\_train} & Receives status of tracks from sensor controller \\
    \texttt{rx\_bells(state)}        & \texttt{state=on, off}          & Receives state of bells from bells controller    \\ \hline
    \texttt{turnon\_lights}          &                                 & Turns on lights                                  \\
    \texttt{turnoff\_lights}         &                                 & Turns off lights                                 \\
    \texttt{tx\_Lights(state)}       & \texttt{state=on, off}          & Transmits state of lights to bells controller    \\ \hline
    \end{tabular}
\caption{Interactions of the lights controller}
\label{tab:lightsSignals}
\end{table}

In \cref{tab:lightsSignals} the interactions of the lights controller with the other controllers and the lights are shown.
Most of the interactions are communications with the other controllers with the actions \textit{turnon\_lights} and \textit{turnoff\_lights} being responsible for actually controlling the lights.

\section{Bell Controller}
\begin{table}[H]
\centering
    \begin{tabular}{|l|l|l|}
    \hline
    \texttt{rx\_sensors\_agg(Train)} & \texttt{Train=train, no\_train} & Receives status of tracks from sensor controller                     \\
    \texttt{rx\_lights(state)}       & \texttt{state=on, off}          & Receives state of lights from lights controller                      \\
    \texttt{rx\_barriers(state)}     & \texttt{state=on, off}          & Receives state of barriers from barriers controller                  \\ \hline
    \texttt{turnon\_bells}           &                                 & Turns bells on                                                       \\
    \texttt{turnoff\_bells}          &                                 & Turns bells off                                                      \\
    \texttt{tx\_bells(state)}        & \texttt{state=on, off}          & Transmits state of bells to bells controller and barriers controller \\ \hline
    \end{tabular}
\caption{Interactions of the bells controller}
\label{tab:bellsSignals}
\end{table}

The interactions of the bells controller shown in \cref{tab:bellsSignals} look a lot like those of the lights controller and require no further explanation.

\section{Barrier Controller}
\begin{table}[H]
\centering
    \begin{tabular}{|l|l|}
    \hline
    \texttt{rx\_sensors\_agg(Train)} & \texttt{Train=train, no\_train}                  \\
                                     & Receives status of tracks from sensor controller \\
    \texttt{rx\_bells(state)}        & \texttt{state=on, off}                           \\
                                     & Receives state of bells from bells controller    \\ \hline
    \texttt{lower\_barriers}         &                                                  \\
                                     & Lowers the barriers                              \\
    \texttt{raise\_barriers}         &                                                  \\
                                     & Raises the barriers                              \\
    \texttt{tx\_barriers(state)}     & \texttt{state=on, off}                           \\
                                     & Transmits state of barriers to bells controller  \\ \hline
    \end{tabular}
\caption{Interactions of the bells controller}
\label{tab:barriersSignals}
\end{table}

Once again the interactions show in \cref{tab:barriersSignals} follow the same template as earlier discussed controllers and require no further explanation.

\section{Communications}
\begin{table}[H]
\centering
\begin{tabular}{|l|l|l|}
\hline
\texttt{rx\_sensors(Train)|tx\_sensors(Train)}               & \texttt{Train=train, no\_train} & \texttt{comm\_sensors(Train)}      \\
\texttt{rx\_sensors\_agg(Train)|tx\_sensors\_agg(Train)}     & \texttt{Train=train, no\_train} & \texttt{comm\_sensors\_agg(Train)} \\
\texttt{rx\_lights(state)|tx\_lights(state)}                 & \texttt{state=on, off}          & \texttt{comm\_lights(state)}       \\
\texttt{rx\_bells(state)|tx\_bells(state)}                   & \texttt{state=on, off}          & \texttt{comm\_bells(state)}        \\
\texttt{rx\_barriers(state)|tx\_barriers(state)}             & \texttt{state=on, off}          & \texttt{comm\_barriers(state)}      \\ \hline
\end{tabular}
\caption{Communications in the system}
\label{tab:communicationsTable}
\end{table}

\cref{tab:communicationsTable} shows the communications defined in the model.
%TODO[c]: this sentence is useless, since the caption of the tab should already cover this.
\end{document}