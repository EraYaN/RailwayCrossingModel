%!TEX program=xelatex
%!TEX spellcheck = en_GB
\documentclass[final]{report}
\input{../../.library/preamble.tex}
\input{../../.library/style.tex}
\addbibresource{../../.library/bibliography.bib}
\begin{document}
\chapter{Interactions}\label{ch:interactions}
This system implements five parallel components that interact with each other and the environment.
In this section the interactions of the different components are defined per component and divided in input and output signals.

\section{Type Definitions}\label{sec:interactions-typedef}
All the system components make use of a couple type definitions.
The \texttt{sort} structures used throughout the report are described below.

To make a distinction between the $n$ tracks of \cref{fig:system-architecture}, the a \texttt{TrackID} structure with the possible values of \texttt{t1} and \texttt{t2}.
For the three sensors of a track, a similar structure is needed: \texttt{SensorID} with the possible values \texttt{s\_w}, \texttt{s\_m} and \texttt{s\_e}.

Most of the subsystems control some external peripheral which has either an on, or off state.
To model this, the \texttt{State} structure is defined being either \texttt{on} or \texttt{off}.

The last structure that is used throughout the model is \texttt{Train} which is used to indicate whether there is currently a train on the track or not (\texttt{train}, \texttt{no\_train}).

\section{Interactions}\label{sec:interactions-external}
\subsection{Incoming}
Each track has its on set of three sensors, each of the sensors can do \texttt{set\_sensors} action, this actions has three parameters being \texttt{TrackID}, \texttt{SensorID} and \texttt{State}.

\subsection{Outgoing}
The outgoing actions for each of the three controllers are showed in \cref{tab:external-actions}, they are self explanatory.

\begin{table}[H]
    \centering
    \caption{External interactions of the system}
    \label{tab:external-actions}
    \begin{tabular}{lll}
        \toprule
        \textbf{\nameref{sec:architecture-light-controller}} & \textbf{\nameref{sec:architecture-bell-controller}} & \textbf{\nameref{sec:architecture-barrier-controller}} \\ 
        \midrule
        \texttt{turnon\_light1}                     & \texttt{turnon\_bell1}                     & \texttt{lower\_barrier1} \\
        \texttt{turnon\_light2}                     & \texttt{turnon\_bell2}                     & \texttt{lower\_barrier2} \\
        \texttt{turnoff\_light1}                    & \texttt{turnoff\_bell1}                    & \texttt{raise\_barrier1} \\
        \texttt{turnoff\_light2}                    & \texttt{turnoff\_bell2}                    & \texttt{raise\_barrier2} \\ 
        \bottomrule
    \end{tabular}
    
\end{table}

\subsection{Error}
In our model there are also some error states, these are modelled by first doing an \texttt{error} action and then entering the deadlock state by doing the action \texttt{delta}.


\subsection{Internal Communication}
In \cref{tab:internal-actions} all internal actions are shown.
The first section is the action that make sure the sensors can get triggered.
The second section are the action that are used to communicate between the lights, bells, and barriers.
The last two sections are to communicate the sensors values to the aggregator and distribute the signal to the light, bells, and barriers.

\begin{table}[H]
    \centering
    \caption{Internal interactions of the system}
    \label{tab:internal-actions}
    \begin{tabular}{ll}
        \toprule
        \textbf{Action} & \textbf{Data} \\ 
        \midrule
        %Helper actions
        set\_sensors & TrackID, SensorID, State \\
        \midrule
        %Communication actions
        tx\_lights & State \\
        rx\_lights & State \\
        comm\_lights & State \\
        tx\_bells\_to\_lights & State \\
        rx\_bells\_to\_lights & State \\
        comm\_bells\_to\_lights & State \\
        tx\_bells\_to\_barriers & State \\
        rx\_bells\_to\_barriers & State \\
        comm\_bells\_to\_barriers & State \\
        tx\_barriers & State \\
        rx\_barriers & State \\
        comm\_barriers & State \\
        \midrule
        tx\_sensors & TrackID, Train \\
        rx\_sensors & TrackID, Train \\
        comm\_sensors & TrackID, Train \\
        \midrule
        tx\_sensors\_agg\_to\_lights & Train \\
        rx\_sensors\_agg\_to\_lights & Train \\
        comm\_sensors\_agg\_to\_lights & Train \\
        tx\_sensors\_agg\_to\_bells & Train \\
        rx\_sensors\_agg\_to\_bells & Train \\
        comm\_sensors\_agg\_to\_bells & Train \\
        tx\_sensors\_agg\_to\_barriers & Train \\
        rx\_sensors\_agg\_to\_barriers & Train \\
        comm\_sensors\_agg\_to\_barriers & Train \\
        \midrule
        error & \\
        \bottomrule
    \end{tabular}
    


\end{table}

\end{document}