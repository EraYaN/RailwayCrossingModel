%!TEX program = xelatex
%!TEX spellcheck = en_GB
\documentclass[final]{report}
\input{../../.library/preamble.tex}
\input{../../.library/style.tex}
\addbibresource{../../.library/bibliography.bib}
\begin{document}
\chapter{Requirements}
In this section, we will gather and describe the requirements of our railway crossing system needed for the later design, modelling and validation.
First of all, we will enumerate and name the components of our system; and then describe the sequence in which they actuate in a fixed situation.
Finally, we will establish the requirements our system has to fulfil for its modelling.

\section{System Components}
The system has to control or interface with the components listed below.
\begin{itemize}
\item 2 Train Tracks ($T_{1}$, $T_{2}$)
\item 2 Crossing Roads ($R_{1}$, $R_{2}$)
\item 2 Barriers ($Ba_{1}$, $Ba_{2}$)
\item 2 Pairs of Flashing Lights ($L_{1}$, $L_{2}$)
\item 2 Bells ($Be_{1}$, $Be_{2}$)
\item 6 Train Sensors ($S_{1,w}$, $S_{1,m}$, $S_{1,e}$, $S_{2,w}$, $S_{2,m}$, $S_{2,e}$)
\end{itemize}

\section{System Configuration}
Below a scetch shows the basic configuration of all the system components and their position.
\begin{figure}[H]
	\centering
	\subimport{resources/}{system-configuration.tikz}
	\caption{System Configuration}
	\label{fig:system-configuration}
\end{figure}

\section{System Requirements}\label{sec:requirements}plan
The most obvious requirement is of course that the system has to prevent cars from driving on the crossing when there is a train coming.

This means that the barriers have to be closed before a train enters the crossing and that they have to stay closed until the crossing is clear of trains.

Besides this trivial requirement, there are a couple other functional requirements for the system.

\begin{enumerate}
\item The system has to be able to handle trains arriving on both tracks from both directions.
	\begin{enumerate}
	\item When a train left the crossing and the barriers are opening again, a new train can retrigger the closing sequence again, which has priority over the opening sequence.
	\end{enumerate}

\item The beams, lights and bells on both sides of the track operate synchronously.
\end{enumerate}

\subsection{Definitions}
Lights ON means the actual lights are blinking but Lights is just a continuous signal for the hardware controller that is responsible for the blinking.
%TODO[TD]:'Lights is just a continuous signal'your lights controller should only support actions, however you can assume this action is picked up by some external controller, which will handle the blinking

A train `on the track' means that a train is somewhere, either partially or completely, between the two outer train sensors of a train track.

\subsection{Sensors}
	\begin{enumerate}
		\item The Sensors node reads the three train sensors on a track.
		%TODO[TD]:sensors are not pollable, but generate actions
		\item The Sensors node decides whether a train is on the track or not.
		%TODO[TD]:is this a requirement?
		\item The Sensors node indicates whether a train is on either of the tracks.
	\end{enumerate}

\subsection{Light}
	\begin{enumerate}
		\item Lights always have to be on on when a train is on the tracks.
		\item Lights always have to be on on when the bell is on.
		\item Lights may only stop blinking when the barriers are open and the bell is off and no train is on the tracks.
	\end{enumerate}

\subsection{Bells}
	\begin{enumerate}
		\item Bells have to turn on a set (to be determined) time after the lights turn on.
		%TODO[TD]:time should not be modeled in this course
		\item Bells may only stop ringing when the barriers are open and no train is on the tracks.
	\end{enumerate}

\subsection{Barriers}
	\begin{enumerate}
		\item Barriers always have to be closed when a train is on the tracks and both the lights and bells are on.
		\item Barriers have to close a set (to be determined) time after the bells start ringing.
		\item Barriers may only open when there is no train on the tracks.
	\end{enumerate}


\section{Sequence}
According to the \nameref{sec:requirements}, two sequences can be described for the system.
One will be for opening the barriers when a train is arriving at the crossing and the other one is for opening the barriers when the train passed by.

\subsection{Barriers closing}
\begin{enumerate}
\item Lights start blinking
\item Wait a few blinks, so no one is scared by an instant bell
		%TODO[TD]:time should not be modeled in this course
\item Bells start ringing
\item Wait for all the cars to drive off of the intersection and give them time to brake
\item Barriers close
\end{enumerate}

\subsection{Barriers opening}
\begin{enumerate}
\item Barriers open
\item Bells stops ringing
\item Lights stop blinking
\end{enumerate}


\section{Possible extensions}
This is of course a basic system, leaving some room for possible extensions in the future.
A few of the possibilities that we already discussed are listed below.

	\begin{enumerate}
	\item Fault detection, recovery and signaling:
		\begin{enumerate}
			\item Bells: Detect if the Lights acted as they are supposed to and if not, recover the error and signal a operation fault.
			\item Barriers: Detect if the Bells acted as they are supposed to and if not, recover the error and signal a operation fault.
			\item Barriers: Detect if Barriers are really closed and if not, signal an operation fault.
		\end{enumerate}

	\item Extend the system with a Car detector to detect if all cars left the crossing.
	\item Extend the system with a Signal light for trains to indicate if the crossing is safe to enter.
	\item Adding an Sensor closer to the crossing so the system sooner knows that the tracks are clear.
	\item Double the output signals of each controller and add user access signal, so that if any output signal shows a discrepancy such as being the lights on and off at the same time, it can be manually corrected.
	\end{enumerate}
\end{document}