%!TEX program = xelatex
%!TEX spellcheck = en_GB
\documentclass[final]{report}
% Include all project wide packages here.
%\usepackage{fullpage}
\usepackage[a4paper,margin=2.5cm,top=2cm]{geometry}
\usepackage{polyglossia}
\setmainlanguage{english}
\usepackage{csquotes}
\usepackage{graphicx}
\usepackage{pdfpages}
\usepackage{caption}
\usepackage[list=true]{subcaption}
\usepackage{float}
\usepackage{standalone}
\usepackage{import}
\usepackage{tocloft}
\usepackage{wrapfig}
\usepackage{authblk}
\usepackage{array}
\usepackage{booktabs}
\usepackage[title,titletoc]{appendix}
\usepackage{fontspec}
\usepackage{pgfplots}
\usepackage{tikz}
\usepackage{siunitx}
\usepackage{units}
\usepackage{amsmath}
\usepackage{mathtools}
\usepackage{unicode-math}
\usepackage{rotating}
\usepackage{titlesec}
\usepackage{titletoc}
\usepackage{blindtext}
\usepackage{color}
\usepackage{enumitem}
\usepackage{tabularx}
\usepackage{titling}
\usepackage[%
siunitx,
fulldiodes,
europeanvoltages,
europeancurrents,
europeanresistors,
americaninductors,
smartlabels]{circuitikz}

\usetikzlibrary{calc}
\usetikzlibrary{positioning}

\pgfplotsset{compat=newest}
\pgfplotsset{plot coordinates/math parser=false}
\usetikzlibrary{plotmarks}
\usepgfplotslibrary{patchplots}
\newlength\figureheight
\newlength\figurewidth

\usepackage[
%backend=bibtex,
backend=biber,
	texencoding=utf8,
bibencoding=utf8,
style=numeric,
citestyle=numeric,
    sortlocale=en_US,
    language=auto,
    backref=true,
    abbreviate=false,
    date=iso8601
]{biblatex}


\usepackage{listings}
\newcommand{\includecode}[4][c]{\lstinputlisting[caption=#2, escapechar=, style=#1,label=#4]{#3}}
\newcommand{\superscript}[1]{\ensuremath{^{\textrm{#1}}}}
\newcommand{\subscript}[1]{\ensuremath{_{\textrm{#1}}}}


\newcommand{\chapternumber}{\thechapter}
\renewcommand{\appendixname}{Appendix}
\renewcommand{\appendixtocname}{Appendices}
\renewcommand{\appendixpagename}{Appendices}


\setlist[enumerate]{labelsep=*, leftmargin=1.5pc}
\setlist[enumerate,1]{label=\arabic*., ref=\arabic*}
\setlist[enumerate,2]{label=\arabic*.,ref=\theenumi.\arabic*}
\setlist[enumerate,3]{label=\arabic*., ref=\theenumii.\arabic*}

\usepackage{xr-hyper}
\usepackage[hidelinks]{hyperref} %<--------ALTIJD ALS LAATSTE
\usepackage[nameinlink,noabbrev,capitalise]{cleveref} %<------- Clever Ref moet na hyperref
\crefname{app}{Appendix}{Appendices}
%\renewcommand{\familydefault}{\sfdefault}


\setmainfont{Myriad Pro}[Ligatures={Common,TeX}]
%\setmathfont{Asana Math}
\setmathfont{Asana-Math.otf}
\setmonofont[Scale=0.9]{Lucida Console}
\newfontfamily\headingfont{Minion Pro}[Ligatures={Common,TeX}]


%Design colors
\definecolor{accent1}{RGB}{0,100,200}
\definecolor{accent2}{RGB}{0,50,100}
\definecolor{tu-cyan}{RGB}{0,166,214}

\newcommand{\hsp}{\hspace{20pt}}
\titleformat{\chapter}[hang]{\Huge\headingfont}{\chapternumber\hsp\textcolor{accent2}{|}\hsp}{0pt}{\Huge\headingfont}

\titleformat{name=\chapter,numberless}[hang]{\Huge\headingfont}{\hsp\textcolor{accent2}{|}\hsp}{0pt}{\Huge\headingfont}

\titleformat{\section}[block]{\LARGE\headingfont}{\arabic{chapter}.\arabic{section}}{0.4em}{}
\titleformat{\subsection}[block]{\Large\headingfont}{\arabic{chapter}.\arabic{section}.\arabic{subsection}}{0.4em}{}
\titleformat{\subsubsection}[block]{\large\headingfont}{\arabic{chapter}.\arabic{section}.\arabic{subsection}.\arabic{subsubsection}}{0.4em}{}
\renewcommand{\arraystretch}{1.2}
\renewcommand{\baselinestretch}{1.25} 

\renewcommand\cfttoctitlefont{\headingfont\Huge}
\renewcommand\cftloftitlefont{\headingfont\Huge}
\renewcommand\cftlottitlefont{\headingfont\Huge}
\setcounter{lofdepth}{2}
\setcounter{lotdepth}{2}


\setlength{\parindent}{0pt}
\setlength{\parskip}{1em}


%SIuntix settings:
%default: 0V to 10V
%custom: 0 - 10V
\sisetup{range-phrase=--}
\sisetup{range-units=single}
\DeclareSIUnit\years{years}

%For code listings
\definecolor{black}{rgb}{0,0,0}
\definecolor{browntags}{rgb}{0.65,0.1,0.1}
\definecolor{bluestrings}{rgb}{0,0,1}
\definecolor{graycomments}{rgb}{0.4,0.4,0.4}
\definecolor{redkeywords}{rgb}{1,0,0}
\definecolor{bluekeywords}{rgb}{0.13,0.13,0.8}
\definecolor{greencomments}{rgb}{0,0.5,0}
\definecolor{redstrings}{rgb}{0.9,0,0}
\definecolor{purpleidentifiers}{rgb}{0.01,0,0.01}


\lstdefinestyle{csharp}{
language=[Sharp]C,
showspaces=false,
showtabs=false,
breaklines=true,
showstringspaces=false,
breakatwhitespace=true,
escapeinside={(*@}{@*)},
columns=fullflexible,
commentstyle=\color{greencomments},
keywordstyle=\color{bluekeywords}\bfseries,
stringstyle=\color{redstrings},
identifierstyle=\color{purpleidentifiers},
basicstyle=\ttfamily\small}

\lstdefinestyle{c}{
language=C,
showspaces=false,
showtabs=false,
breaklines=true,
showstringspaces=false,
breakatwhitespace=true,
escapeinside={(*@}{@*)},
columns=fullflexible,
commentstyle=\color{greencomments},
keywordstyle=\color{bluekeywords}\bfseries,
stringstyle=\color{redstrings},
identifierstyle=\color{purpleidentifiers},
}

\lstdefinestyle{matlab}{
language=Matlab,
showspaces=false,
showtabs=false,
breaklines=true,
showstringspaces=false,
breakatwhitespace=true,
escapeinside={(*@}{@*)},
columns=fullflexible,
commentstyle=\color{greencomments},
keywordstyle=\color{bluekeywords}\bfseries,
stringstyle=\color{redstrings},
identifierstyle=\color{purpleidentifiers}
}

\lstdefinestyle{vhdl}{
language=VHDL,
showspaces=false,
showtabs=false,
breaklines=true,
showstringspaces=false,
breakatwhitespace=true,
escapeinside={(*@}{@*)},
columns=fullflexible,
commentstyle=\color{greencomments},
keywordstyle=\color{bluekeywords}\bfseries,
stringstyle=\color{redstrings},
identifierstyle=\color{purpleidentifiers}
}

\lstdefinestyle{xaml}{
language=XML,
showspaces=false,
showtabs=false,
breaklines=true,
showstringspaces=false,
breakatwhitespace=true,
escapeinside={(*@}{@*)},
columns=fullflexible,
commentstyle=\color{greencomments},
keywordstyle=\color{redkeywords},
stringstyle=\color{bluestrings},
tagstyle=\color{browntags},
morestring=[b]",
  morecomment=[s]{<?}{?>},
  morekeywords={xmlns,version,typex:AsyncRecords,x:Arguments,x:Boolean,x:Byte,x:Char,x:Class,x:ClassAttributes,x:ClassModifier,x:Code,x:ConnectionId,x:Decimal,x:Double,x:FactoryMethod,x:FieldModifier,x:Int16,x:Int32,x:Int64,x:Key,x:Members,x:Name,x:Object,x:Property,x:Shared,x:Single,x:String,x:Subclass,x:SynchronousMode,x:TimeSpan,x:TypeArguments,x:Uid,x:Uri,x:XData,Grid.Column,Grid.ColumnSpan,Click,ClipToBounds,Content,DropDownOpened,FontSize,Foreground,Header,Height,HorizontalAlignment,HorizontalContentAlignment,IsCancel,IsDefault,IsEnabled,IsSelected,Margin,MinHeight,MinWidth,Padding,SnapsToDevicePixels,Target,TextWrapping,Title,VerticalAlignment,VerticalContentAlignment,Width,WindowStartupLocation,Binding,Mode,OneWay,xmlns:x}
}

\lstdefinestyle{python}{
language=Python,
showspaces=false,
showtabs=false,
breaklines=true,
showstringspaces=false,
breakatwhitespace=true,
escapeinside={(*@}{@*)},
columns=fullflexible,
commentstyle=\color{greencomments},
keywordstyle=\color{bluekeywords}\bfseries,
stringstyle=\color{redstrings},
identifierstyle=\color{purpleidentifiers},
}

%defaults
\lstset{
basicstyle=\ttfamily\scriptsize ,
extendedchars=false,
numbers=left,
numberstyle=\ttfamily\tiny,
stepnumber=1,
tabsize=4,
numbersep=5pt
}
\addbibresource{../../.library/bibliography.bib}
\begin{document}
\chapter{Requirements}
In this section the requirements for the railway crossing system will be gathered and described.
First of all, we will enumerate and name the components of our system; and then describe the sequence in which they actuate in a fixed situation.
Finally, we will establish the requirements our system has to fulfil for its modelling.

\section{System Components}
The system has to control or interface with the components listed below.
\begin{itemize}
\item 2 Train Tracks ($T_{1}$, $T_{2}$)
\item 2 Crossing Roads ($R_{1}$, $R_{2}$)
\item 2 Barriers ($Ba_{1}$, $Ba_{2}$)
\item 2 Pairs of Flashing Lights ($L_{1}$, $L_{2}$)
\item 2 Bells ($Be_{1}$, $Be_{2}$)
\item 6 Train Sensors ($S_{1,w}$, $S_{1,m}$, $S_{1,e}$, $S_{2,w}$, $S_{2,m}$, $S_{2,e}$)
\end{itemize}

\section{System Configuration}
Below a sketch shows the basic configuration of all the system components and their position.
\begin{figure}[H]
	\centering
	\subimport{resources/}{system-configuration.tikz}
	\caption{System Configuration}
	\label{fig:system-configuration}
\end{figure}

\section{System Functionality}\label{sec:requirements}
The function of the system is of course preventing cars from driving on the crossing when there is a train coming.
This means that the barriers have to be closed with the lights and bells being on before a train enters the crossing and that they have to stay closed until the crossing is clear of trains.
There are two other functions that the system needs to be able to fulfil:

\begin{enumerate}
\item The system has to be able to handle trains arriving on both tracks from both directions.
	\begin{enumerate}
	\item When a train left the crossing and the barriers are opening again, a new train can re-trigger the closing sequence again, which has priority over the opening sequence.	
	\end{enumerate}	

\item The barriers, lights and bells on each side of the track will operate in parallel but following the sequence of operation that will be explained in the following section.
\end{enumerate}

The first one will be expressed further on as a general requirement or liveness requirement for the system.
Here it will be considered that the system can handle trains moving in both track when the requirements that will be enunciated  later on are verified for both tracks.
In other words, the normal sequence of operation can be triggered when any train comes in any track.

The second one will be implemented in the model with different multiactions.	



\subsection{Sequence}
Two sequences can be described for the system.
One will be for closing the crossing when a train needs to pass.
The other one is for opening the crossing when the train has passed.

Crossing closing:
\begin{enumerate}
\item Lights start blinking
\item Bells start ringing
\item Barriers close
\end{enumerate}

Crossing opening:
\begin{enumerate}
\item Barriers open
\item Bells stop ringing
\item Lights stop blinking
\end{enumerate}

\section{Requirements defined}
Firstly, we will give some definitions that may help to the description and understanding of the formulated requirements.
A train `on the track' means that a train is somewhere, either partially or completely, between the two outer train sensors of a train track.
A train is `approaching' when it triggers one of the external sensors for the first time, while it is `exiting' the track when it triggers the other external sensor.
When the middle sensor is triggered, we consider the track `occupied'.
As for the transition states, a train is `traversing' when it is moving from `approaching' to `occupied' and it is `leaving' when it is moving between `occupied' and `exit' states.\\
%Regarding the lights, \texttt{Lights ON} means the actual lights are blinking but the Light Controller only gives an action to an external blinking controller which is not part of this project scope.

\subsection{Sensors}
These requirements should be considered either for a train that it is moving from West to East and from East to West.
The sensors are considered to generate an action whenever their state has suffered a change, that way, we will only be able to know if a sensor is on, if its previous state was off and vice versa.
Whenever an output signal from the sensor controller stating a train has been detected should be sent, we will refer it as a \texttt{train} signal, while we will refer a \texttt{no\_train} signal otherwise.
The requirements for the sensors are the following:
%TODO[jeroen]:These first six requirements are not verifiable
%TODO[jeroen]:I would expect everything up to this point to be part of the introduction.
%TODO[c]: this whole part is old anyways (based on the old shitty model), I think we have to rewrite this whole part, and make it short as if it could be an introductionary part-thingy
	\begin{enumerate}
		\item There cannot be a \texttt{no\_train} signal after a train has triggered an ON in any outer sensor for the first time (i.e a train has entered the tracks) and has not triggered an 
an OFF in the middle sensor.
		\item There cannot be a \texttt{train} signal after a train has triggered an OFF in the middle sensor and has not triggered an OFF in any outer sensor (i.e. a train is leaving the
tracks).
		\item There cannot be a \texttt{train} signal if after a train has triggered an OFF in any outer sensor and not triggered an ON in the middle sensor (i.e a train has exited
 the tracks); another train does not trigger an ON in any outer sensor.		
	\end{enumerate}

\subsection{Lights}
	\begin{enumerate}
		\item The lights cannot turn on when there is no train on the tracks.
		\item The lights cannot turn off as long as there is a train on the tracks.
		\item The lights cannot turn off as long as the bells are on.
	\end{enumerate}

\subsection{Bells}
	\begin{enumerate}
		\item The bells cannot turn on if the lights are off.
		\item The bells cannot turn on if there is no train on the track.
		\item The bells cannot turn off if the barriers are lowered.
		\item The bells cannot turn off if there is a train on the track.
	\end{enumerate}

\subsection{Barriers}
	\begin{enumerate}
		\item The barriers cannot be raised when there is a train on the tracks.
		\item The barriers cannot be lowered if there is no train on the tracks.
		\item The barriers cannot be lowered if the bells are off.
	\end{enumerate}
	
\subsection{General}
	\begin{enumerate}
	\item From any state, the crossing needs to be able to be closed (barriers lowered, lights and bells on).
	\item From any state, the crossing needs to be able to be opened (lights and bells off, barriers raised).
	\end{enumerate}


\section{Possible extensions}
This is of course a basic system, leaving some room for possible extensions in the future.
A few of the possibilities that we already discussed are listed below.

	\begin{enumerate}
	\item Fault detection, recovery and signalling:
		\begin{enumerate}
			\item Bells: Detect if the Lights acted as they are supposed to and if not, recover the error and signal a operation fault.
			\item Barriers: Detect if the Bells acted as they are supposed to and if not, recover the error and signal a operation fault.
			\item Barriers: Detect if Barriers are really closed and if not, signal an operation fault.
		\end{enumerate}

	\item Extend the system with a Car detector to detect if all cars left the crossing.
	\item Extend the system with a Signal light for trains to indicate if the crossing is safe to enter.
	\item Adding an Sensor closer to the crossing so the system sooner knows that the tracks are clear.
	\item Double the output signals of each controller and add user access signal, so that if any output signal shows a discrepancy such as being the lights on and off at the same time, it can be manually corrected.
	\end{enumerate}
\end{document}