%!TEX program = xelatex
%!TEX spellcheck = en_GB
\documentclass[final]{report}
\input{../../.library/preamble.tex}
\input{../../.library/style.tex}
\addbibresource{../../.library/bibliography.bib}
\begin{document}
\chapter{Introduction}
An embedded systems engineer developing a product puts a lot of effort in delivering a solution that works.
But when can you say with certainty that a product works? Of course, once a system does what it is supposed to do one can call it day and ship the product.
But can you be certain that it can do what is supposed to do, all of the time, without fail? There are plenty of embedded systems in this world where a failure to function or unexpected behaviour could have disastrous consequences.
This is where the science of system validation comes in to play.

In this project our team sets out to model the control of a typical Dutch railway crossing.
Trains can come from both directions and are detected by sensors on the tracks.
Once a train is detected a system of warning lights, bells and of course barriers ensures that no car is on the crossing when a train needs to pass.
The system will consist of at least four parallel components and none of those is allowed to be a god controller that knows the full state of every subsystem.
One can imagine that a fault in such a system could be fatal, so it is a good example of a scenario where improper validation is unacceptable.

In this report the final result of the project will be discussed.
First the system will be described in detail with attention especially being paid to the requirements of the system formulated in natural language.
The basic architecture of the system will be discussed and the interactions of the different components will be defined.
After that the requirements will be formulated using $\mu$-calculus taught in the lectures and a description will be given of the mCRL2 model and it's design process during the course of the project.
In the last part of the report the verification process will be discussed that resulted in all requirements being verified giving us a validated system.
\end{document}