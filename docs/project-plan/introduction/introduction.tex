%!TEX program = xelatex
%!TEX spellcheck = en_GB
\documentclass[final]{report}
\input{../../.library/preamble.tex}
\input{../../.library/style.tex}
\addbibresource{../../.library/bibliography.bib}
\begin{document}
\chapter{Introduction}
When an embedded systems engineer is working on a project he is always working towards a product that works.
But when can you say with certainty that a product works? Of course, once a system does what it is supposed to do one can call it day and ship the product.
But can you be certain that it can do what is supposed to do, all of the time, without fail? There are plenty of embedded systems in this world where a failure to function or unexpected behaviour could have disastrous consequences.
This is where the science of system validation comes in to play.

In this project our team sets out to model the control of a typical Dutch railway crossing.
Trains can come from both directions and are detected by sensors on the tracks.
Once a train is detected a system of warning lights, bells and of course barriers are supposed to ensure no car is on the crossing when a train needs to pass.
One important requirement is that the system needs to consist of at least 4 parallel components.
The system is not allowed to have a so called god controller, an object knowing the full state of every subsystem.
Besides these non functional requirements, a set of functional requirements will be proposed. A model of this system is designed. Once the model is completed, it needs to be validated to ensure that all the requirements are met.
One can imagine that a fault in such a system could be fatal, so it is a good example of a scenario where improper validation is unacceptable.

In this second deliverable at least a first iteration is done on every part of the system.
Attempts at formulating $\mu$-calculus formulae for the requirements have been made and the model has seen significant progress.
\end{document}