%!TEX program = xelatex
%!TEX spellcheck = en_GB
\documentclass[final]{report}
\input{../../.library/preamble.tex}
\input{../../.library/style.tex}
\addbibresource{../../.library/bibliography.bib}
\begin{document}
\chapter{Verification}
\label{ch:verification}
All the μ-calculus formulas from \cref{ch:translated-requirements} were put in a *.pre.mcf file.
This is basically a \textbackslash n\&\&\textbackslash n separated file.
Our build script splits this into many *.mcf files and runs the lts2pbes and pbes2bool tools on them.
Then it collects all the results and either displays a lot of green to the user or a lot of red.
Before the verification takes place the model is built using mcrl22lps and lps2lts.
There is also the possibility to generate all the traces of a certain action using lps2lts and tracepp.
All this is implemented in python 3.5; tested on Windows 7, Windows 10, Linux mint 17.3; using the latest stable mCRL2.


\end{document}