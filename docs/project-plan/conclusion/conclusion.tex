%!TEX program = xelatex
%!TEX spellcheck = en_GB
\documentclass[final]{report}
\input{../../.library/preamble.tex}
\input{../../.library/style.tex}
\addbibresource{../../.library/bibliography.bib}
\begin{document}
\chapter{Conclusion}
In conclusion it can definitely be stated that the project was a success. 
As described in \cref{ch:verification} all requirements formulated in \cref{ch:translated-requirements} could be verified for the model. 
Of course, getting to that result was not a straightforward effort. 
In the first phases of the project there was great difficulty in getting started on the model or the requirements because a lot of the knowledge needed for that was still to be discussed in the lectures. 
Most notably this resulted in a set of requirements in the first deliverable that turned out to be unverifiable later. A lot of time was also spent on trying to get a grip on the workings of mCRL2 in the first hours of the project. 
It required a way of thinking quite different from programming that would normally be encountered in work or study. 
After considerable effort and time spent a lot of the ill fated early attempts at the model or the requirements were corrected making for a final result that satisfies all requirements. 
Not only the requirements formulated for a correct functioning of the system but also the requirements stated by the project itself in regards to minimal amount of parallel controllers and absence of a god controller.
\end{document}